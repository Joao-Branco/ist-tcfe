\newpage
\section{Simulation Analysis}
\label{sec:simulation}

\subsection{AC/DC converter}
\label{sec:AC/DC converter}

At the beginning an AC/DC converter circuit was developed.The circuit used is in Figure~\ref{fig:circuit}. In Point 1, an AC/DC converter was simulated for 10 periods. The circuit as four types of components which are 2 resistances, 1 capacitor, 23 diodes and a transformer. 



At the second step (Point 2 of simulation), the average voltage was measured  using 1000 points. In the table below (\ref{tab:op_MAXMIN_tab}) can be seen the values of the ripple, the maximum, the minimum and the average for the voltage.
 
To accomplish next point (Point 3), the output voltage ripple was determined by subtracting the maximum and minimum values of the voltage. The values of the ripple are represented in table \ref{tab:op_MAXMIN_tab}. These value are represented by ymax-ymin.

\begin{table}[h]
  \centering
  \begin{tabular}{|l|r|}
    \hline    
    {\bf Values of y} & {\bf Voltage [V]} \\ \hline
    \input{op_MAXMIN_tab.tex}
  \end{tabular}
  \caption{Voltage values}
  \label{tab:op_MAXMIN_tab}
\end{table}




Next the voltages at Envelope Detector and Voltage Regulator circuits were plotted by using the corresponding node. Both of these plots can be seen in the figures below. (Figure \ref{fig:envelope} and Figure \ref{fig:voltageregulator}).
Because the values of the resistors and the capacitor were chosen with the ideal values for our circuit it is difficult to notice the variation in the graph of Figure \ref{fig:envelope}.  

\begin{figure}[h] \centering
\includegraphics[width=0.5\linewidth]{envelope.pdf}
\caption{Voltages at Envelope Detector}
\label{fig:envelope}
\end{figure}

\begin{figure}[h] \centering
\includegraphics[width=0.4\linewidth]{voltageregulator.pdf}
\caption{Voltages at Voltage Regulator}
\label{fig:voltageregulator}
\end{figure}


\newpage


The final step of the simulation analyses was to find the difference between the voltage obtained and 12 Volts that was the desired value. The plot can be seen in Figure \ref{fig:ripple}.

\begin{figure}[h] \centering
\includegraphics[width=0.4\linewidth]{ripple.pdf}\caption{Output voltage ripple}
\label{fig:ripple}
\end{figure}




In the plot is possible to recognise that initially this difference is almost -12 V because the voltage obtained starts in zero and stabilises at 12 V. For these reason, the difference as time goes by tends to zero.




