\newpage
\section{Simulation Analysis}
\label{sec:simulation}

\subsection{Operating Point Analysis}
\label{sec:Operating}
At the first step (Point 1), considering t<0, a DC circuit was obtained. The branch that contained the capacitor is in open circuit, so the current is zero. Making this approach was possible to determine the voltages in all nodes and the currents in all branches, that can be seen in Tables ~\ref{tab:op_volt1_tab} and ~\ref{tab:op_amp1_tab}.

\begin{table}[h]
  \centering
  \begin{tabular}{|l|r|}
    \hline    
    {\bf Node} & {\bf Voltage [V]} \\ \hline
    \input{op_volt1_tab.tex}
  \end{tabular}
  \caption{Initial Voltage at the nodes}
  \label{tab:op_volt1_tab}
\end{table}

\begin{table}[h]
  \centering
  \begin{tabular}{|l|r|}
    \hline    
    {\bf Branch} & {\bf Current [A]} \\ \hline
    \input{op_amp1_tab.tex}
  \end{tabular}
  \caption{Initial currents in all branches}
  \label{tab:op_amp1_tab}
\end{table}


\newpage
At the second step (Point 2 of simulation), considering t=0, $V_s$(0)=0 and the capacitor is replaced by $V_x$ = V(6) - V(8). V(6) and V(8) were obtained through point 1. This procedure is necessary because $Req$ is needed to calculate the natural solution, and the only way  to do that is to use a voltage source instead of a capacitor.  
Voltages at the nodes and currents at the brunches are presented respectively in Tables ~\ref{tab:op_volt2_tab} and ~\ref{tab:op_amp2_tab}. 

\begin{table}[h]
  \centering
  \begin{tabular}{|l|r|}
    \hline    
    {\bf Node} & {\bf Voltage [V]} \\ \hline
    \input{op_volt2_tab.tex}
  \end{tabular}
  \caption{Voltage at the nodes with voltage source}
  \label{tab:op_volt2_tab}
\end{table}

\begin{table}[h]
  \centering
  \begin{tabular}{|l|r|}
    \hline    
    {\bf Branch} & {\bf Current [A]} \\ \hline
    \input{op_amp2_tab.tex}
  \end{tabular}
  \caption{Currents in all branches with voltage source}
  \label{tab:op_amp2_tab}
\end{table}




\newpage
\subsection{Transient Analysis}

To accomplish next point, using the transient analysis was possible to obtain $V_6$(t) between [0;20] (ms). In order to do this plot, natural response of the circuit was used considering boundary conditions in $V_6$ and $V_8$, that were obtain at \ref{sec:Operating}. This plot can be seen in Figure \ref{fig:trans1}.

\begin{figure}[h] \centering
\includegraphics[width=0.5\linewidth]{trans1.pdf}
\caption{Transient Analysis between 0 and 20 ms}
\label{fig:trans1}
\end{figure}


Next it was simulated the natural and forced response of $V_6$(t) by repeating the previous point but considering alternate current, $V_s$(t), with frequency f=1kHz.
Figure \ref{fig:trans2} shows both responses - natural and forced.

\begin{figure}[h] \centering
\includegraphics[width=0.5\linewidth]{trans2.pdf}
\caption{Transient Analysis of $V_s$ and $V_6$ between 0 and 20 ms}
\label{fig:trans2}
\end{figure}


\newpage
\subsection{Frequency Analysis}

\subsubsection{Magnitude and Phase Response - Bode Diagram}

To accomplish this analysis, frequency response (magnitude and phase) at node 6 was simulated - $V_6$(f). With this response plots were created together with $V_s$(f).  Magnitude response is represented in Figure ~\ref{fig:acm} and frequency response in Figure ~\ref{fig:acp}.

\begin{figure}[h] \centering
\includegraphics[width=0.5\linewidth]{acm.pdf}
\caption{Magnitude diagram between 0.1 and $10^7$}
\label{fig:acm}
\end{figure}

\begin{figure}[h] \centering
\includegraphics[width=0.5\linewidth]{acp.pdf}
\caption{Phase diagram between 0.1 and $10^7$}
\label{fig:acp}
\end{figure}



When frequency goes to infinity, impedance of the capacitor tends to zero (short circuit). So, if impedance tends to zero, resistance tends to zero, and therefore there is not voltage drop and $V_c$ tends to zero. That's the reason of the negative slope  of the magnitude.








\newpage
\section{Results Analysis}
\label{sec:resultsanalysis}

Comparing the results in all points:

Analysing initial values, theoretical analysis ~\ref{tab:oct2_tab} and simulation ~\ref{tab:op_volt1_tab} converge perfectly (5 decimal place).

When the capacitor was substituted by a voltage source with the same voltage drop, the both results to node 6 are the same (8.74800 V) and the currents at R5 and $V_x$ (simulation) and $I_x$ (theoretical) - -0.00278897 A. 

In transient analysis (natural, forced and the combination of both), it is possible to observe that the graphics between 0 and 20 ms have the same plot in both analyses. 

In figures 5 (theoretical) and 9 (simulation) and it can be observe that $V_s$ and $V_6$ are in phases opposition, how it was expected from the theoretical knowledge.

In the last point that were approached in this laboratory report, the frequency analysis, we can observe that the results are similarities and the functions in order to frequency has the same behaviour.