\section{Theoretical Analysis}
\label{sec:analysis}

In this section, the circuit shown in Figure~\ref{fig:circuit} is analysed
theoretically, in terms nodes voltages. Values given by $Octave$ are compiled in table \ref{tab:oct1_tab} for Nodes Method.

\begin{table}[h]
  \centering
  \begin{tabular}{|l|r|}
    \hline    
    {\bf Node} & {\bf Voltage [V]} \\ \hline
    \input{oct1_tab.tex}
  \end{tabular}
  \caption{Initial voltage values on the nodes}
  \label{tab:oct1_tab}
\end{table}



\subsection{Operating Point Analysis}

The Nodes Method consists on the analysis of the nodes that connect components in this circuit. Besides the previous said, it's not possible to apply the method to nodes were voltage sources are connected ($V_a$ and $V_c$). To overcome this problem, additional equations for nodes related by voltage sources were used, combined with the \textit{Supernode Method} (used in nodes 5 and 8) that consists in combining two nodes. 

The circuit used to write the following equations is presented in Figure \ref{fig:nodes}.



\begin{equation}
  (V_1 - V_2) G_1 - (V_2 - V_5) G_3 + (V_3 - V_2) G_2 = 0
  \label{eq_1}
\end{equation}

\begin{equation}
  (V_5 - V_6) G_5 - K_b (V_2 - V_5) = 0
  \label{eq_2}
\end{equation}

\begin{equation}
  - (V_3 - V_2) G_2 + K_b (V_2 - V_5) = 0
  \label{eq_3}
\end{equation}

\begin{equation}
  - (V_7 - V_8) G_7 - V_7 . G_6 = 0
  \label{eq_4}
\end{equation}

\begin{equation}
  V_1 = V_s
  \label{eq_5}
\end{equation}

\begin{equation}
  V_5 - V_8 + K_d . V_7 . G_6 = 0
  \label{eq_6}
\end{equation}

\begin{equation}
  - (V_5 - V_6) G_5 + (V_2 - V_5) G_3 - V_5 . G_4 + (V_7 - V_8) G_7 = 0
  \label{eq_7}
\end{equation}


\begin{figure}[h] \centering
\includegraphics[width=0.5\linewidth]{nodes.pdf}
\caption{Current Direction}
\label{fig:nodes}
\end{figure}


\newpage


Additional equations for nodes related by voltage sources were used, combined with the \textit{Supernode Method} (used in nodes 5, 6 e 8) that consists in combining three nodes.

The circuit used to write the following equations is presented in Figure \ref{fig:nodes}. In that circuit, the capacitor were replaced by a voltage source $V_x$ corresponding to the voltage difference at nodes 6 and 8.

On this procedure, the capacitor was replaced by a voltage source. This procedure is necessary because $Req$ is needed to calculate the natural solution, and the only way  to do that is to use a voltage source instead of a capacitor.  

The voltage at the nodes is shown in table \ref{tab:oct2_tab}, also the values of $I_x$, $Req$ and Tau are given.


\begin{equation}
  (V_1 - V_2) G_1 - (V_2 - V_5) G_3 + (V_3 - V_2) G_2 = 0
  \label{eq_8}
\end{equation}

\begin{equation}
  (V_6 - V_8) = V_x
  \label{eq_9}
\end{equation}

\begin{equation}
  - (V_3 - V_2) G_2 + K_b (V_2 - V_5) = 0
  \label{eq_10}
\end{equation}

\begin{equation}
  - (V_7 - V_8) G_7 - V_7 . G_6 = 0
  \label{eq_11}
\end{equation}

\begin{equation}
  V_1 = V_s
  \label{eq_12}
\end{equation}

\begin{equation}
  V_5 - V_8 + K_d . V_7 . G_6 = 0
  \label{eq_13}
\end{equation}

\begin{equation}
  (V_2 - V_5) G_3 - V_5 . G_4 + (V_7 - V_8) G_7 - Kb * (V_2 - V_5) = 0
  \label{eq_14}
\end{equation}


\begin{table}[h]
  \centering
  \begin{tabular}{|l|r|}
    \hline    
    {\bf Node} & {\bf Voltage, Current and Resister [V, A, Ohm]} \\ \hline
    \input{oct2_tab.tex}
  \end{tabular}
  \caption{Voltage values on the nodes}
  \label{tab:oct2_tab}
\end{table}

\newpage
\subsection{Transient Analysis}

\subsubsection{Natural Solution}

With the results obtained in the previous point, it was possible to determine the natural solution for $V_6n$. The graph for this solution is shown in figure \ref{fig:natural} and the equation for the same is:


\begin{equation}
  V_n(t) = V_x . e ^ \frac{-t}{R_eq * C}
  \label{eq_15}
\end{equation}


\begin{figure}[h] \centering
\includegraphics[width=0.5\linewidth]{natural.eps}
\caption{Natural Response from 0 to 20 ms}
\label{fig:natural}
\end{figure}


\subsubsection{Forced Solution}

In order to calculate the forced solution for $V_6f$, we used the suggestion and ran the nodal analysis. We obtained a complex value for $V_6$ and real values for the other voltages. Using the complex value was possible to calculate the amplitude and phase used to compute the forced solution. Figure \ref{fig:forced} represents the forced solution and table \ref{tab:solveNos_tab} shows the complex amplitudes in the nodes. 

\begin{table}[h]
  \centering
  \begin{tabular}{|l|r|}
    \hline    
    {\bf Node} & {\bf Voltage [V]} \\ \hline
    \input{solveNos_tab.tex}
  \end{tabular}
  \caption{Complex amplitudes in the nodes}
  \label{tab:solveNos_tab}
\end{table}

\begin{figure}[h] \centering
\includegraphics[width=0.5\linewidth]{forced.eps}
\caption{Forced Response from 0 to 20 ms}
\label{fig:forced}
\end{figure}

\begin{equation}
    Gain = 0.56801 
    \label{}
\end{equation}

\begin{equation}
    Phase = - 2.9980 (rad)
    \label{}
\end{equation}

\begin{equation}
    V_6f = Gain \times sin (2 * \pi * 1000 * t + Phase)
    \label{}
\end{equation}


\subsubsection{Comparison between $V_6$ and $V_s$ }

In order to find the final total solution $V_6(t)$, natural and forced solutions were superimposed. The equations bellow show the the final total solution for $V_6(t)$, and $V_s(t)$ is given in the procedure description. The figure \ref{fig:all} represents both in the interval [-5;20] (ms).

\begin{figure}[h] \centering
\includegraphics[width=0.5\linewidth]{all.eps}
\caption{Comparison between $V_6$ and $V_s$  from -5 to 20 ms}
\label{fig:all}
\end{figure}

\begin{equation}
    V_s (t) = V_s ,   t<0
    \label{}
\end{equation}

\begin{equation}
    V_s (t) = sin(2 * \pi * 100 * t) , t>=0
    \label{}
\end{equation}

\begin{equation}
    V_6 (t) = 5.7986,   t<0
    \label{}
\end{equation}

\begin{equation}
    V_6 (t) = V_x . e ^ \frac{-t}{R_eq * C} + Gain * sin(2 * \pi * 100 * t + Phase) , t>=0
    \label{}
\end{equation}



\subsection{Frequency Analysis}

The objective of this point is to study the frequency response of the nodes 6 and 8, and their difference. To accomplish this, magnitude and phase diagrams were used. Domain of equations obtains in previous points was changed, by an iterative method, in order to have the frequency response.
In Figures \ref{fig:bodemag} and \ref{fig:bodepha} are represented magnitude and phase response, respectively.

\begin{figure}[h] \centering
\includegraphics[width=0.5\linewidth]{bodemag.eps}
\caption{Magnitude Diagram between 0.1 and $10^6$ decades}
\label{fig:bodemag}
\end{figure}

\begin{figure}[h] \centering
\includegraphics[width=0.5\linewidth]{bodepha.eps}
\caption{Phase Diagram between 0.1 and $10^6$ decades}
\label{fig:bodepha}
\end{figure}

\newpage
