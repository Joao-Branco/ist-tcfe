\newpage
\section{Theoretical Analysis}
\label{sec:analysis}

In this section, a suitable theoretical model to predict the output of the Envelope Detector and Voltage Regulator circuits was used.


After obtaining in \textit{Octave} the required model for the Envelope Detector and Voltage Regulator, points 1 to 5 were repeated with the goal to compare the simulation with the theoretical model.

\subsection{Envelope Detector}

Initially it was needed to low the voltage, for that a transformer was used with n=0.73896. This can be observed in Figure \ref{fig:inputWave}.  

\begin{figure}[h] \centering
\includegraphics[width=0.5\linewidth]{inputWave.eps}
\caption{Transformer used in this circuit}
\label{fig:inputWave}
\end{figure}

\newpage

After defining the new voltage obtained in the transformer it was needed a Full wave rectifier. The Full wave rectifier has four diodes that lead the circuit to a 0.7 V voltage drop.

Combining the Full wave rectifier with a resistance and a capacitor in parallel it was possible to have an Envelope Detector. 

The voltage after the Envelope Detector its given by the following equations. 

\begin{equation}
    v_{0}= v_s ,  D_{on}
\end{equation} 

\begin{equation}
    v_{0}= - Ri_C ,   D_{off}
\end{equation} 


In order to solve the previous equations it is needed the instant where the Diodes are on or off. To find these values the following equation was solved:

\begin{equation}
    i_R= - i_c 
\end{equation}

The final expression for $t_{off}$ is:

\begin{equation}
   t_{off}=\frac{1}{\omega } \arctan (-CR\omega ) + \frac{k}{2f} 
\end{equation}

k belong to natural numbers.

The final expression for the voltage with $t_{off}$ is:

\begin{equation}
  v_0(t)=A\sin (\omega*t_{off})e^{-\frac{t-t_{off}}{RC}} 
\end{equation}

With $t_{on}$, the expression for the voltage is:

\begin{equation}
    v_{0}= v_s
\end{equation}

After computing these parameters the plot showed in Figure \ref{fig:venvlope}  was obtained.

\begin{figure}[ht] \centering
\includegraphics[width=0.7\linewidth]{venvlope.eps}
\caption{Envelope Detector}
\label{fig:venvlope}
\end{figure}


\subsection{Voltage Regulator}

In this part an incremental analysis was made where the diode model is approximately to a resistor. The formula used is:

\begin{equation}
    i_d \approx v_d / r_d
    \label{eq:id}
\end{equation}

Using this approximation (Equation \ref{eq:id}) is possible to plot a graph that represents the final output voltage. This plot is shown in Figure \ref{fig:Vregulator}.

\begin{figure}[h] \centering
\includegraphics[width=0.6\linewidth]{Vregulator.eps}
\caption{Voltage Regulator}
\label{fig:Vregulator}
\end{figure}

Besides a theoretical value of the ripple were obtained. The following table (table \ref{tab:ripple_tab}) shows this value.

\begin{table}[h]
  \centering
  \begin{tabular}{|l|r|}
    \hline    
    {\bf Values of y} & {\bf Voltage [V]} \\ \hline
    \input{ripple_tab}
  \end{tabular}
  \caption{Voltage values from theoretical analyses}
  \label{tab:ripple_tab}
\end{table}

\newpage

\section{Results Analysis}
\label{sec:resultsanalysis}

In the presented section, the mail goal is to compare side by side the results obtained in the simulation with \textit{Ngspice} and the theoretical analysis obtained with \textit{Octave}. Two comparisons will be made. One for the Envelope Detector and other for the Voltage Regulator circuit. 


In Figure \ref{fig:point1} we can compare the results obtained for the Envelope Detector for both methods. 

\begin{figure}[h]
\centering
  \begin{subfigure}[b]{0.4\textwidth}
  \centering
    \includegraphics[scale = 0.4]{venvlope.eps}
  \end{subfigure}
  \hfill
  \begin{subfigure}[b]{0.4\textwidth}
  \centering
    \includegraphics[scale = 0.3]{envelope.pdf}
  \end{subfigure}
  \caption{Envelope Detector}
  \label{fig:point1}
\end{figure}


Analysing both images in Figure \ref{fig:point1} it is possible to observe that both the simulation and the theoretical analyses have similar values for the voltage. This value stabilises at 170V.

\newpage

Now a comparison between the theoretical values and the simulation for the voltage regulator is made. In the previous tables is possible to compare both the results.

The difference between the values is expected, since in the theoretical analysis the ideal diode model was used and in the simulation the default model was used. 




\begin{table}
\begin{center}
\begin{tabular}{|c|c|}
 \hline    
    {\bf Values of y} & {\bf Voltage [V]} \\ \hline
    \input{op_MAXMIN_tab.tex}
\end{tabular}
\begin{tabular}{|c|c|}
\hline    
    {\bf Values of y} & {\bf Voltage [V]} \\ \hline
    \input{ripple_tab}
\end{tabular}
\caption{Comparison between theoretical and simulation values}
\end{center}
\end{table}



The value found for the merit is calculated with the following expression:

\begin{equation}
  M= \frac{1}{cost * (ripple(v_0)+average(v_0)-12)+10^{-6})}
  \label{eq:merit}
\end{equation}

Where the cost is calculated with a $R_1$ = 300k$\Omega$, $R_3$ = 100k$ \Omega$, C = 150 $\mu$F and 22 Diodes.
Thus, the cost give us 552.2 MU.

Using equation \ref{eq:merit}, the merit is 1.708.


















