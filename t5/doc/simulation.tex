\section{Simulation Analysis}
\label{sec:simulation}

\textit{Ngspice} does not allow the use of capacitors in parallel and in series so in order to use the values of the capacitors that are given in utterance and to achieve 330 $\mu$F, finding the equivalent capacitor was necessary. For that each of the three capacitors used in the calculation of the equivalent capacitor have the value of 220 $\mu$F.

The following equation shows the calculation of the equivalent capacitor.

\begin{equation}
    C_{eq}= \frac{1}{C_{1}}+\frac{1}{C_{2}+C_{3}} 
\label{eq:vin}
\end{equation} 

\subsection{Frequency Response}
\label{sec:Frequency Response}

In this section, the circuit in analysis was developed in \textit{Ngspice}. The first goal, was to plot gain and phase frequency response analysis, as it can be seen in Figures \ref{fig:gfr} and \ref{fig:pfr}.

\begin{figure}[h]
  \centering
  \begin{minipage}[b]{0.36\textwidth}
    \includegraphics[width=1\textwidth]{Vo1db.pdf}
    \caption{Gain in Frequency Response.}
    \label{fig:gfr}
  \end{minipage}
  \hfill
  \begin{minipage}[b]{0.36\textwidth}
    \includegraphics[width=1\textwidth]{Vo1f.pdf}
    \caption{Phase in Frequency Response.}
    \label{fig:pfr}
  \end{minipage}
\end{figure}

\newpage
From the plot, it can be mentioned that the maximum value for the gain in 1KHz happens for almost 40dB, as required. 

The values obtained for low frequency, high frequency and the gain at 1KHz were:

\begin{table}[h]
\centering
\begin{tabular}{|l|c|}
\hline
\multicolumn{1}{|c|}{} & Value         \\ \hline
Low Frequency          & 105.0228 (Hz) \\ \hline
High Frequency         & 10117.16 (Hz) \\ \hline
Gain                   & 99.87576     \\ \hline
Gain (dB)              & 39.98920 (dB)  \\ \hline
\end{tabular}
\caption{Obtained values from \textit{Ngspice}}
\end{table}


\subsection{Transient Analysis at 1KHz}

To accomplish the Transient analysis at 1KHz, an AC current with the expression \ref{eq:vin} was used:

\begin{equation}
    v_{in}(t) = 0.01 * sin ( 2\pi * 1000 * t)   [V]
\label{eq:vin}
\end{equation} 

\begin{figure}[h] \centering
\includegraphics[width=0.5\linewidth]{Vo1.pdf}
\caption{AC Voltage in time domain.}
\label{fig:envelope}
\end{figure}
\newpage
The previous image (\ref{fig:envelope}) represents the plot of $v_{in}(t)$.

\subsection{Input and Output Impedance Simulation}

In order to obtain the input impedance, a resistance with a high value (infinite) was inserted in parallel with the circuit in analysis. Through out this simulation was possible to determine the voltage and current at the positive node of the OpAmp. With that information, the transfer function was found and the value of the input impedance at 1KHz was calculated. The variation of the input impedance with frequency is shown in Figure \ref{fig:vimpedance_in}.

\begin{figure}[h] \centering
\includegraphics[width=0.5\linewidth]{vimpedance_in.pdf}
\caption{Variation of input impedance with frequency.}
\label{fig:vimpedance_in}
\end{figure}

\newpage

\begin{table}[h]
 \centering
  \begin{tabular}{|l|c|}
   \hline
   \multicolumn{1}{|c|}{} & Module ($\Omega$) \\ \hline
    Input Impedance        & 996289.2                        \\ \hline
    Output Impedance       & 5.038123                       \\ \hline
\end{tabular}
\end{table}

\vspace{0.5cm}
To obtain the output impedance, the $v_{in}$ voltage was removed and added a $v_{out}$. This was done in order not to have current at the entrance of the circuit and to have it at the way out. Through this, it was possible to do all the calculations. The variation of the output impedance with frequency is presented in Figure \ref{fig:vimpedance_out}.

\begin{figure}[h] \centering
  \includegraphics[width=0.5\linewidth]{vimpedance_out.pdf}
    \caption{Variation of output impedance with frequency.}
     \label{fig:vimpedance_out}
\end{figure}

\newpage
\subsection{Operational Amplifier Simulation}

It was necessary to simulate the Bode Diagrams of the OpAmp without any components, as can be seen in figures \ref{fig:ampop_db} and \ref{fig:ampop_f}. With this step, it is possible do find the transfer function of the non-ideal OpAmp, wich is needed at section \ref{sec:analysis}. By analysing the graph was possible do see were the poles of the transfer function were ($s=10^4$ and $s=10^6$).


\begin{figure}[h]
  \centering
  \begin{minipage}[b]{0.4\textwidth}
    \includegraphics[width=1\textwidth]{ampop_db.pdf}
    \caption{Module of the transfer function in frequency response.}
    \label{fig:ampop_db}
  \end{minipage}
  \hfill
  \begin{minipage}[b]{0.4\textwidth}
    \includegraphics[width=1\textwidth]{ampop_f.pdf}
    \caption{Phase of the transfer function in frequency response.}
    \label{fig:ampop_f}
  \end{minipage}
\end{figure}

\begin{equation}
    T(s)=\frac{1*10^6}{(s+1*10^4)(s+1*10^6)}
    \label{eq:tfsim}
\end{equation}
